	Амплитудную решетку можно представить в виде непрозрачного экрана, в котором прорезано большое число $N$ параллельных щелей --- штрихов. Постоянство расстояний между штрихами $d$ и ширина штриха $b$ должны выдерживаться с большой точностью. При наблюдении спектра амплитуда и интенсивность световой волны определяются углом $\varphi$ между нормалью к решетке и направлением дифрагировавших лучей. Будем считать, что амплитуды всех волн одинаковы, т.е. фиксирована амплитуда падающей волны и постоянна площадь всех штрихов. Интенсивность дифрагированного света для углов $\varphi_m$, при которых волны, приходящие в точку наблюдения от всех щелей оказываются в фазе:

	\begin{equation}\label{eq:1}
		d\sin\varphi_m = m\lambda
	\end{equation}

	Величина $m$ называется порядком спектра.

	Угловая дисперсия $D(\lambda)$ характеризует угловое рассеяние между близкими спектральными линиями:

	\begin{equation}\label{eq:2}
		D(\lambda) = \frac{\dif\varphi}{\dif\lambda}
	\end{equation}

	Для угловой дисперсии решетки получаем:

	\begin{equation}\label{eq:3}
		D(\lambda) = \frac{\dif\varphi}{\dif\lambda} = \frac{m}{\sqrt{d^2-m^2\lambda^2}}
	\end{equation}

	C увеличением порядка спектра угловая дисперсия будет возрастать. Для малых углов дифракции угловая дисперсия пропорциональна порядку спектра: $D\simeq m/d$.

